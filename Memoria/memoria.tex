%Compilar: pdflatex -synctex=1 -interaction=nonstopmode --shell-escape apuntesed.tex 
\documentclass[10pt,a4paper,spanish]{article}

\usepackage[spanish]{babel}
\usepackage[utf8]{inputenc}
\usepackage{amsmath, amsthm}
\usepackage{amsfonts, amssymb, latexsym}
\usepackage{enumerate}
\usepackage[official]{eurosym}
\usepackage{graphicx}
\usepackage{graphics}
\usepackage[usenames, dvipsnames]{color}
\usepackage{colortbl}
\usepackage{fancyhdr}
\usepackage{fancybox}
\usepackage{pseudocode}
\usepackage[all]{xy}
%\usepackage{minted}
%\usepackage{tikz}
\usepackage{pgfplots}
\usepackage{multirow}
\usepackage{float}
\usepackage{subfigure}
\pgfplotsset{compat=1.5}

% a4large.sty -- fill an A4 (210mm x 297mm) page
% Note: 1 inch = 25.4 mm = 72.27 pt
%       1 pt = 3.5 mm (approx)

% vertical page layout -- one inch margin top and bottom
\topmargin      0 mm    % top margin less 1 inch
\headheight     0 mm    % height of box containing the head
\headsep        10 mm    % space between the head and the body of the page
\textheight     250 mm
\footskip       14 mm    % distance from bottom of body to bottom of foot

\definecolor{verde}{rgb}{0.0, 0.65, 0.31}

\usepackage[bookmarks=true,
            bookmarksnumbered=false, % true means bookmarks in
                                     % left window are numbered
            bookmarksopen=false,     % true means only level 1
                                     % are displayed.
            colorlinks=true,
            linkcolor=webblue]{hyperref}
\definecolor{webgreen}{rgb}{0, 0.5, 0} % less intense green
\definecolor{webblue}{rgb}{0, 0, 0.5}  % less intense blue
\definecolor{webred}{rgb}{0.5, 0, 0}   % less intense red

\newcommand{\HRule}{\rule{\linewidth}{0.5mm}} % regla horizontal para  el titulo

\usepackage[familydefault,light]{Chivo} %% Option 'familydefault' only if the base font of the document is to be sans serif
\usepackage[T1]{fontenc}

\pagestyle{fancy}
%con esto nos aseguramos de que las cabeceras de capítulo y de sección vayan en minúsculas

% \renewcommand{\chaptermark}[1]{%
%       \markboth{#1}{}}
\renewcommand{\sectionmark}[1]{%
      \markright{\thesection\ #1}}
\fancyhf{} %borra cabecera y pie actuales
\fancyhead[LE,RO]{\textcolor{verde}{\bfseries\thepage}}
\fancyhead[LO]{\bfseries\rightmark}
\renewcommand{\headrulewidth}{0.5pt}
\renewcommand{\footrulewidth}{0pt}
\addtolength{\headheight}{0.5pt} %espacio para la raya
\fancypagestyle{plain}{%
      \fancyhead{} %elimina cabeceras en páginas "plain"
      \renewcommand{\headrulewidth}{0pt} %así como la raya
}

%%%%% Para cambiar el tipo de letra en el título de la sección %%%%%%%%%%%
\usepackage{sectsty}
% \chapterfont{\fontfamily{pag}\selectfont} %% for chapter if you want
\sectionfont{\fontfamily{pag}\selectfont}
\subsectionfont{\fontfamily{pag}\selectfont}
\subsubsectionfont{\fontfamily{pag}\selectfont}


%Definimos autor y título
\title{\bf \textcolor{verde}{Reproductor Spotify con Gestos}}
\author{Marta Gómez Macías y Braulio Vargas López}


\setlength{\parindent}{0pt}
\setlength{\parskip}{1ex plus 0.5ex minus 0.2ex}

\begin{document}
\maketitle

\tableofcontents

\section{\textcolor{verde}El reproductor}
El reproductor usado para la práctica es uno de los ejemplos de código que da la API de \textit{\textcolor{verde}{Spotify}}. El código original se encuentra en \href{https://github.com/possan/webapi-player-example}{Github} y además, también puede probarse \href{http://lab.possan.se/thirtify/#/}{online}. El \textit{\textcolor{verde}{fork}} que hemos hecho de ese repositorio para añadir la interacción con \textit{\textcolor{verde}{Leap Motion}} se encuentra también en \href{https://github.com/BraulioV/webapi-player-example}{Github}.

Básicamente, este reproductor incluye casi todas las cosas que pueden hacerse con la API de \textit{\textcolor{verde}{Spotify}}:

\begin{enumerate}[\qquad \color{verde}{$\bullet$}]
  \item Seleccionar una \textit{\textcolor{verde}{playlists}} del usuario o una de las que hace \textit{\textcolor{verde}{Spotify}}.
  \item Una vez seleccionada una \textit{\textcolor{verde}{playlist}}, reproducir todas las canciones seguidas o seleccionar una de ellas. Debido a limitaciones de la API, \textcolor{verde}{\textbf{sólo pueden reproducirse 30 segundos}} de canción.
  \item Permite controlar el reproductor con botones de \textbf{\textcolor{verde}{Play}}, \textbf{\textcolor{verde}{Pause}}, \textbf{\textcolor{verde}{Next song}}, \textbf{\textcolor{verde}{Previous song}} y \textbf{\textcolor{verde}{controlar el nivel de volumen}} mediante una barra deslizador.
  \item \textbf{\textcolor{verde}{No}} permite la creación de \textit{\textcolor{verde}{playlists}}.
\end{enumerate}

\section{\textcolor{verde}Cómo hemos incluido Leap Motion en el reproductor}
El reproductor está hecho con \textit{\textcolor{verde}{AngularJS}}, un framework de \textit{\textcolor{verde}{Javascript}} que implementa el modelo de Vista-Controlador. En el fichero \texttt{index.html} deben especificarse los controladores que se encargarán de las distintas partes de la aplicación. En el caso del reproductor, el controlador se encuentra en el fichero \texttt{controllers/player.js} y se denomina \texttt{PlayerController}. Como nosotros queríamos integrar los gestos de \textit{\textcolor{verde}{Leap Motion}} en el reproductor, hicimos el código del \texttt{leap.loop} en el propio controlador. También intentamos hacerlo en un controlador a parte, pero no conseguimos que funcionase.

\section{\textcolor{verde}Guía de uso}

Gracias a \textit{\textcolor{verde}{Leap Motion}}, podemos realizar las siguientes acciones a partir de gestos:

\begin{enumerate}
  \item \textbf{\textcolor{verde}{Pausa.}}
  \item \textbf{\textcolor{verde}{Reproducir.}}
  \item \textbf{\textcolor{verde}{Pasar a la siguiente canción.}}
  \item \textbf{\textcolor{verde}{Retroceder a la canción anterior.}}
  \item \textbf{\textcolor{verde}{Subir el volumen.}}
  \item \textbf{\textcolor{verde}{Bajar el volumen.}}
  \item \textbf{\textcolor{verde}{Silenciar (Mute).}}
  \item \textbf{\textcolor{verde}{Recuperar el volumen.}}
  \item \textbf{\textcolor{verde}{Seleccionar y reproducir listas de canciones.}}
\end{enumerate}

A continuación, podemos ver cómo realizar cada uno de los gestos anteriormente vistos.

\subsection{\textcolor{verde}Pausa}

Cuando hay una canción reproduciéndose, podemos pausarla haciendo uso del gesto predefinido en Leap ``\textbf{\textcolor{verde}{keytap}}'' que consiste en realizar un gesto similar a la pulsación de una tecla en el teclado (\hyperref[kt1]{Figura \ref*{kt1}}). También se puede realizar con la mano extendida hacia delante y realizar un golpe seco de muñeca hacia abajo (\hyperref[kt2]{Figura \ref*{kt2}}). Al realizarlos, la canción se pausará y cambiará el icono de reproduciendo a pausado, en la parte inferior izquierda de la pantalla.

En la \hyperref[kt]{Figura \ref*{kt}} podemos ver un ejemplo visual de estos gestos.

\begin{figure}[H]
    \centering
    \mbox {
        
        \subfigure[Gesto con el dedo.]{
            \includegraphics[width=0.5\textwidth]{images/keytap1}
            \label{kt1}
        }
        
        \quad
        
        \subfigure[Gesto con la mano también admitido.] {
            \includegraphics[width=0.35\textwidth]{images/keytap2}
            \label{kt2}
        }
    }
    
    \caption{Gestos para pausar y reproducir la música.}
    \label{kt}
\end{figure}

\subsection{\textcolor{verde}Reproducir}

Este evento se activa cuando la canción que se estaba reproduciendo \textit{\textcolor{verde}{está pausada}}. Para poder volver a reproducir la canción, tenemos que hacer el mismo gesto que para pausar una canción. Estos gestos los podemos ver en la \hyperref[kt]{Figura \ref*{kt}}.

Cuando se lance este evento, el icono de pausa de la parte inferior izquierda de la pantalla cambiará al icono de reproducción. 

\subsection{\textcolor{verde}Pasar a la siguiente canción o a la anterior.}

Estos gestos son bastante similares entre sí, ya que ambos utilizan el gesto predefinido ``\textbf{\textit{\textcolor{verde}{swipe}}}''. Este gesto consiste en pasar la mano desde un lado hacia el lado opuesto pasando por encima del \textit{\textcolor{verde}{Leap}}.

Dependiendo del sentido en el pasemos la mano, podremos pasar o retroceder de canción:

\begin{enumerate}[$\bullet$]
  \item \textbf{\textcolor{verde}{De derecha a izquierda}} Este gesto lo que hace es retroceder una canción en la lista de reproducción activa y comienza a reproducir la canción anterior.
  \item \textbf{\textcolor{verde}{De izquierda a derecha}}: Este sin embargo, hace lo contrario al gesto anterior, pasa a la siguiente canción y comienza a reproducirla.
\end{enumerate}

%%% Insertar aquí las imágenes de la mano haciendo el swipe

\begin{figure}[H]
    \centering
    \mbox {
        
        \subfigure[Gesto para pasar a la siguiente canción.]{
            \includegraphics[width=0.35\textwidth]{images/swipe1}
            \label{sw1}
        }
        
        \quad
        
        \subfigure[Gesto pasa pasar a la canción anterior.] {
            \includegraphics[width=0.35\textwidth]{images/swipe2}
            \label{sw2}
        }
    }
    
    \caption{Gestos para pasar a la siguiente y a la anterior canción.}
    \label{sw}
\end{figure}

\subsection{\textcolor{verde}Subir y bajar el volumen}

Para poder cambiar el volumen levemente, es decir, subirlo o bajarlo un poco, podemos hacer un gesto con el dedo que imita el gesto que haríamos subiendo o bajando el volumen en la rueda de volumen de cualquier radio. A continuación, podemos ver más claro cada gesto:

\begin{enumerate}[$\bullet$]
  \item \textbf{\textcolor{verde}{Bajar el volumen}}: para bajar el volumen, con el dedo estirado hacia la pantalla, podemos girarlo en el sentido antihorario. Girando el dedo en este sentido, reduciremos el volumen un 10\%, hasta un máximo del 5\% del volumen original. Podemos verlo en la \hyperref[anticlkwise]{Figura \ref*{anticlkwise}}.
  \item \textbf{\textcolor{verde}{Subir el volumen}}: este gesto es totalmente contrario al anterior. Se realiza en el sentido horario y subirá el volumen un 5\% hasta llegar al máximo. Podemos verlo en la \hyperref[clkwise]{Figura \ref*{clkwise}}.
\end{enumerate}

Durante la ejecución de ambos gestos, podemos ver cómo la barra de volumen de la parte izquierda de la pantalla va cambiando de forma dinámica según sea el nivel de volumen que haya.

%% Insertar aquí la imagen del dedo girando

\begin{figure}[H]
    \centering
    \mbox {
        
        \subfigure[Girando en sentido antihorario bajamos el volumen.]{
            \includegraphics[width=0.5\textwidth]{images/circle1}
            \label{anticlkwise}
        }
        
        \quad
        
        \subfigure[Girando en sentido horario subimos el volumen.] {
            \includegraphics[width=0.5\textwidth]{images/circle2}
            \label{clkwise}
        }
    }
    
    \caption{Gestos para el volumen del reproductor.}
    \label{dd}
\end{figure}

\subsection{\textcolor{verde}Silenciar y recuperar el sonido}

Estos gestos se hacen con una sola mano, y son para silenciar el volumen (ponerlo a 0) y recuperar el volumen original en caso de que hayamos silenciado la música antes.

Para silenciar la música, lo único que tenemos que hacer es, cerrar el puño para que se silencie la música. Una vez silenciada, la barra de volumen se irá a 0 para indicarnos de que el volumen del reproductor es 0.

Si queremos recuperar el sonido, sólo tenemos que abrir la mano cuando el volumen de la música sea 0, y se recuperará automáticamente el volumen que había antes de silenciar el reproductor.

%%% Insertar aquí las imágenes de los gestos


\subsection{\textcolor{verde}Seleccionar y reproducir listas de canciones}

Este gesto se puede realizar de forma dinámica y se utiliza para señalar qué lista de reproducción de las que aparecen en pantalla queremos reproducir, y una vez dentro de la lista seleccionada, podemos iniciar la reproducción de esta de forma similar a cómo la hemos seleccionado.

Para ello, sólo tenemos que apuntar con dos dedos estirados hacia la pantalla, tal y como se ve en la \hyperref[point]{Figura \ref*{point}}. Cuando lo hagamos, aparecerá un puntero de color azul que nos indica hacia dónde estámos apuntando en la pantalla. Según lo cerca o lejos que estemos de esta, crecerá o encogerá su tamaño.

Para seleccionar una lista de reproducción lo único que tenemos que hacer es mantener el puntero de color azul durante unos segundos encima de la lista de reproducción que queremos reproducir. Pasado este tiempo, se abrirá en el navegador la lista de reproducción y podremos reproducirla manteniendo el cursor unos segundos encima del botón \texttt{Reproducir} que aparece en verde en la pantalla del navegador. 

\begin{figure}[H]
    \centering     
    \includegraphics[width=0.5\textwidth]{images/point}
    \caption{Gesto para seleccionar en la pantalla.}
    \label{point}
\end{figure}

\end{document}
